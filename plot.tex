\documentclass{article}
\usepackage[right=2cm, left=2cm]{geometry}

\usepackage{amsmath}
\usepackage{amsthm}
\usepackage{tikz,pgfplots,pgfplotstable}
\usepackage{enumitem}
\usepackage{tcolorbox}
\usepackage{piton}

\pgfplotsset{compat=1.18}
\pgfplotsset{compat/labels=pre 1.3}
\usepgfplotslibrary{groupplots}

\renewcommand{\a}{a}
\newcommand{\maxa}{\bar{\a}}
\newcommand{\alength}{\ell}
\newcommand{\vecsize}[1]{n_{#1}}
\newcommand{\iteration}{w}
\newcommand{\aeffect}{\omega}
\newcommand{\pmiss}{\hat{p}_{\text{m}}}


\title{Stealthy Computational Delay Attacks on Control Systems}

\begin{document}
\maketitle

\begin{figure*}
    \centering
    \begin{tikzpicture} 
 \begin{groupplot} [%
 group style={columns=4, rows=2,
   xlabels at=edge bottom, ylabels at=edge left,
   xticklabels at=edge bottom, yticklabels at=edge left,
   horizontal sep=0.5cm, vertical sep=0.5cm},
 yticklabel style = {/pgf/number format/fixed, /pgf/number format/precision=5},
 scaled y ticks=false,
 %
 height=5cm, width = 0.25\textwidth,
 %
 ymin = 0,
 xmin = 1, xmax = 25,
 ylabel style={yshift=.3cm},
 xlabel={$\alength$}, xlabel near ticks,
 grid=major,
 xtick = {5, 10, 15, 20, 25},
 grid style = {densely dashed, black!30},
 legend style={draw = none, at={(1.05,1.05)}, anchor=south, font=\small},
 legend columns = 3]

 % reading files for tanks
 \pgfplotstableread[col sep=comma]{results_bin_chi2/qt_nmp/a1_p0.09_alpha0.01_w20.csv}\attackonenineq
 \pgfplotstableread[col sep=comma]{results_bin_chi2/qt_nmp/a1_p0.17_alpha0.01_w20.csv}\attackoneseventeenq
 \pgfplotstableread[col sep=comma]{results_bin_chi2/qt_nmp/a1_p0.25_alpha0.01_w20.csv}\attackonetwentyfiveq
 \pgfplotstableread[col sep=comma]{results_bin_chi2/qt_nmp/a2_p0.09_alpha0.01_w20.csv}\attacktwonineq
 \pgfplotstableread[col sep=comma]{results_bin_chi2/qt_nmp/a2_p0.17_alpha0.01_w20.csv}\attacktwoseventeenq
 \pgfplotstableread[col sep=comma]{results_bin_chi2/qt_nmp/a2_p0.25_alpha0.01_w20.csv}\attacktwotwentyfiveq
 \pgfplotstableread[col sep=comma]{results_bin_chi2/qt_nmp/a3_p0.09_alpha0.01_w20.csv}\attackthreenineq
 \pgfplotstableread[col sep=comma]{results_bin_chi2/qt_nmp/a3_p0.17_alpha0.01_w20.csv}\attackthreeseventeenq
 \pgfplotstableread[col sep=comma]{results_bin_chi2/qt_nmp/a3_p0.25_alpha0.01_w20.csv}\attackthreetwentyfiveq
 \pgfplotstableread[col sep=comma]{results_bin_chi2/qt_nmp/a4_p0.09_alpha0.01_w20.csv}\attackfournineq
 \pgfplotstableread[col sep=comma]{results_bin_chi2/qt_nmp/a4_p0.17_alpha0.01_w20.csv}\attackfourseventeenq
 \pgfplotstableread[col sep=comma]{results_bin_chi2/qt_nmp/a4_p0.25_alpha0.01_w20.csv}\attackfourtwentyfiveq
 % reading files for pendulum
 \pgfplotstableread[col sep=comma]{results_bin_chi2/furuta_zero/a1_p0.09_alpha0.01_w20.csv}\attackonenine
 \pgfplotstableread[col sep=comma]{results_bin_chi2/furuta_zero/a1_p0.17_alpha0.01_w20.csv}\attackoneseventeen
 \pgfplotstableread[col sep=comma]{results_bin_chi2/furuta_zero/a1_p0.25_alpha0.01_w20.csv}\attackonetwentyfive
 \pgfplotstableread[col sep=comma]{results_bin_chi2/furuta_zero/a2_p0.09_alpha0.01_w20.csv}\attacktwonine
 \pgfplotstableread[col sep=comma]{results_bin_chi2/furuta_zero/a2_p0.17_alpha0.01_w20.csv}\attacktwoseventeen
 \pgfplotstableread[col sep=comma]{results_bin_chi2/furuta_zero/a2_p0.25_alpha0.01_w20.csv}\attacktwotwentyfive
 \pgfplotstableread[col sep=comma]{results_bin_chi2/furuta_zero/a3_p0.09_alpha0.01_w20.csv}\attackthreenine
 \pgfplotstableread[col sep=comma]{results_bin_chi2/furuta_zero/a3_p0.17_alpha0.01_w20.csv}\attackthreeseventeen
 \pgfplotstableread[col sep=comma]{results_bin_chi2/furuta_zero/a3_p0.25_alpha0.01_w20.csv}\attackthreetwentyfive
 \pgfplotstableread[col sep=comma]{results_bin_chi2/furuta_zero/a4_p0.09_alpha0.01_w20.csv}\attackfournine
 \pgfplotstableread[col sep=comma]{results_bin_chi2/furuta_zero/a4_p0.17_alpha0.01_w20.csv}\attackfourseventeen
 \pgfplotstableread[col sep=comma]{results_bin_chi2/furuta_zero/a4_p0.25_alpha0.01_w20.csv}\attackfourtwentyfive
 
 % TANK PROCESS -----------------------------------------------------------------------------------------------------------------------------------
 \nextgroupplot[ylabel style={align=center}, ylabel = {Quadruple tank\\$\aeffect$}, ymax=0.0025]
 \addplot[thick, blue, mark=*]
   table[ x expr=\thisrow{length}-2, restrict x to domain=0:25,y expr=(\thisrow{attack}-\thisrow{nominal})/4] {\attackonenineq};
 \addplot[thick, red, mark=*, mark options={solid, fill=white}]
   table[ x expr=\thisrow{length}-2, restrict x to domain=0:25, y expr=(\thisrow{attack}-\thisrow{nominal})/4]{\attackoneseventeenq};
 \addplot[thick, brown, mark=x]
   table[ x expr=\thisrow{length}-2, restrict x to domain=0:25, y expr=(\thisrow{attack}-\thisrow{nominal})/4]{\attackonetwentyfiveq};
 \node[draw, fill=white, anchor=north east] at (axis cs:25,0.0025) {$\maxa = 1$};

 \nextgroupplot[ymax=0.0025]
 \addplot[thick, blue, mark=*]
   table[ x expr=\thisrow{length}-2,restrict x to domain=0:25,y expr=(\thisrow{attack}-\thisrow{nominal})/4]{\attacktwonineq};
 \addlegendentry{$\pmiss = 0.09$}
 \addplot[thick, red, mark=*, mark options={solid, fill=white}]
   table[ x expr=\thisrow{length}-2,restrict x to domain=0:25,y expr=(\thisrow{attack}-\thisrow{nominal})/4]{\attacktwoseventeenq};
 \addlegendentry{$\pmiss = 0.17$}
 \addplot[thick, brown, mark=x]
   table[ x expr=\thisrow{length}-2,restrict x to domain=0:25,y expr=(\thisrow{attack}-\thisrow{nominal})/4]{\attacktwotwentyfiveq};
 \addlegendentry{$\pmiss = 0.25$}
 \node[draw, fill=white, anchor=north east] at (axis cs:25,0.0025) {$\maxa = 2$};

 \nextgroupplot[ymax=0.0025]
 \addplot[thick, blue, mark=*]
   table[ x expr=\thisrow{length}-2,restrict x to domain=0:25,y expr=(\thisrow{attack}-\thisrow{nominal})/4]{\attackthreenineq};
 \addplot[thick, red, mark=*, mark options={solid, fill=white}]
   table[ x expr=\thisrow{length}-2,restrict x to domain=0:25,y expr=(\thisrow{attack}-\thisrow{nominal})/4]{\attackthreeseventeenq};
 \addplot[thick, brown, mark=x]
   table[ x expr=\thisrow{length}-2,restrict x to domain=0:25,y expr=(\thisrow{attack}-\thisrow{nominal})/4]{\attackthreetwentyfiveq};
 \node[draw, fill=white, anchor=north east] at (axis cs:25,0.0025) {$\maxa = 3$};

 \nextgroupplot[ymax=0.0025]
 \addplot[thick, blue, mark=*]
   table[ x expr=\thisrow{length}-2,restrict x to domain=0:25,y expr=(\thisrow{attack}-\thisrow{nominal})/4]{\attackfournineq};
 \addplot[thick, red, mark=*, mark options={solid, fill=white}]
   table[ x expr=\thisrow{length}-2,restrict x to domain=0:25,y expr=(\thisrow{attack}-\thisrow{nominal})/4]{\attackfourseventeenq};
 \addplot[thick, brown, mark=x]
   table[ x expr=\thisrow{length}-2,restrict x to domain=0:25,y expr=(\thisrow{attack}-\thisrow{nominal})/4]{\attackfourtwentyfiveq};
 \node[draw, fill=white, anchor=north east] at (axis cs:25,0.0025) {$\maxa = 4$};

 % PENDULUM -----------------------------------------------------------------------------------------------------------------------------------
 \nextgroupplot[ylabel style={align=center}, ylabel = {Furuta pendulum\\$\aeffect$}, ymax=0.0001]
 \addplot[thick, blue, mark=*]
   table[ x expr=\thisrow{length}-2, restrict x to domain=0:25, y expr=\thisrow{attack}-\thisrow{nominal}]{\attackonenine};
 \addplot[thick, red, mark=*, mark options={solid, fill=white}]
   table[ x expr=\thisrow{length}-2, restrict x to domain=0:25, y expr=\thisrow{attack}-\thisrow{nominal}]{\attackoneseventeen};
 \addplot[thick, brown, mark=x]
   table[ x expr=\thisrow{length}-2, restrict x to domain=0:25, y expr=\thisrow{attack}-\thisrow{nominal}]{\attackonetwentyfive};
 \node[draw, fill=white, anchor=north east] at (axis cs:25,0.0001) {$\maxa = 1$};
 
 \nextgroupplot[ymax=0.0001]
 \addplot[thick, blue, mark=*]
   table[ x expr=\thisrow{length}-2,restrict x to domain=0:25,y expr=\thisrow{attack}-\thisrow{nominal}]{\attacktwonine};
 \addplot[thick, red, mark=*, mark options={solid, fill=white}]
   table[ x expr=\thisrow{length}-2,restrict x to domain=0:25,y expr=\thisrow{attack}-\thisrow{nominal}]{\attacktwoseventeen};
 \addplot[thick, brown, mark=x]
   table[ x expr=\thisrow{length}-2,restrict x to domain=0:25,y expr=\thisrow{attack}-\thisrow{nominal}]{\attacktwotwentyfive};
 \node[draw, fill=white, anchor=north east] at (axis cs:25,0.0001) {$\maxa = 2$};

 \nextgroupplot[ymax=0.0001]
 \addplot[thick, blue, mark=*]
   table[ x expr=\thisrow{length}-2,restrict x to domain=0:25,y expr=\thisrow{attack}-\thisrow{nominal}]{\attackthreenine};
 \addplot[thick, red, mark=*, mark options={solid, fill=white}]
   table[ x expr=\thisrow{length}-2,restrict x to domain=0:25,y expr=\thisrow{attack}-\thisrow{nominal}]{\attackthreeseventeen};
 \addplot[thick, brown, mark=x]
   table[ x expr=\thisrow{length}-2,restrict x to domain=0:25,y expr=\thisrow{attack}-\thisrow{nominal}]{\attackthreetwentyfive};
 \node[draw, fill=white, anchor=north east] at (axis cs:25,0.0001) {$\maxa = 3$}; 

 \nextgroupplot[ymax=0.0001]
 \addplot[thick, blue, mark=*]
   table[ x expr=\thisrow{length}-2,restrict x to domain=0:25,y expr=\thisrow{attack}-\thisrow{nominal}]{\attackfournine};
 \addplot[thick, red, mark=*, mark options={solid, fill=white}]
   table[ x expr=\thisrow{length}-2,restrict x to domain=0:25,y expr=\thisrow{attack}-\thisrow{nominal}]{\attackfourseventeen};
 \addplot[thick, brown, mark=x]
   table[ x expr=\thisrow{length}-2,restrict x to domain=0:25,y expr=\thisrow{attack}-\thisrow{nominal}]{\attackfourtwentyfive};
 \node[draw, fill=white, anchor=north east] at (axis cs:25,0.0001) {$\maxa = 4$}; 
 
 \end{groupplot}
\end{tikzpicture}

    \vspace{-1cm}
    \caption{Attack effectiveness $\aeffect$ with window size $\vecsize{\iteration} = 20$ and combined binomial and $\chi^2$ detector. The tests are conducted using the hold actuation policy for the quadruple tank and the zero actuation policy for the Furuta pendulum.}
    \vspace{-.35cm}
    \label{fig:effect_maxa}
\end{figure*}

\begin{figure}
    \centering
    \begin{tikzpicture} 
 \begin{groupplot} [%
 group style={columns=2, rows=3,
   xlabels at=edge bottom, ylabels at=edge left,
   xticklabels at=edge bottom,
   horizontal sep=1.25cm, vertical sep=0.5cm},
 yticklabel style = {/pgf/number format/fixed, /pgf/number format/precision=5},
 scaled y ticks=false,
 %
 height=6cm, width = 0.4\textwidth,
 %
 ymin = 0,
 xmin = 1, xmax = 25,
 ylabel style={yshift=.3cm},
 xlabel={$\alength$}, xlabel near ticks,
 grid=major,
 xtick = {5, 10, 15, 20, 25},
 grid style = {densely dashed, black!30},
 legend style={draw = none, at={(-0.225,1.05)}, anchor=south, font=\small},
 legend columns = 3,
 title style={at={(0.5,1.2)}, anchor=south, align=center},
 ylabel = {$\aeffect$}, ylabel near ticks,
 ]

 % reading files for tanks
 \pgfplotstableread[col sep=comma]{results_bin/qt_nmp/a4_p0.09_alpha0.01_w20.csv}\attackbinnine
 \pgfplotstableread[col sep=comma]{results_bin/qt_nmp/a4_p0.17_alpha0.01_w20.csv}\attackbinseventeen
 \pgfplotstableread[col sep=comma]{results_bin/qt_nmp/a4_p0.25_alpha0.01_w20.csv}\attackbintwentyfive
 \pgfplotstableread[col sep=comma]{results_bin_chi2/qt_nmp/a4_p0.09_alpha0.01_w20.csv}\attackbinchinine
 \pgfplotstableread[col sep=comma]{results_bin_chi2/qt_nmp/a4_p0.17_alpha0.01_w20.csv}\attackbinchiseventeen
 \pgfplotstableread[col sep=comma]{results_bin_chi2/qt_nmp/a4_p0.25_alpha0.01_w20.csv}\attackbinchitwentyfive
 \pgfplotstableread[col sep=comma]{results_chi2/qt_nmp/a4_p0.09_alpha0.01_w20.csv}\attackchinine
 \pgfplotstableread[col sep=comma]{results_chi2/qt_nmp/a4_p0.17_alpha0.01_w20.csv}\attackchiseventeen
 \pgfplotstableread[col sep=comma]{results_chi2/qt_nmp/a4_p0.25_alpha0.01_w20.csv}\attackchitwentyfive
 
 % reading files for pendulum
 \pgfplotstableread[col sep=comma]{results_bin/furuta_zero/a4_p0.09_alpha0.01_w20.csv}\fzattackbinnine
 \pgfplotstableread[col sep=comma]{results_bin/furuta_zero/a4_p0.17_alpha0.01_w20.csv}\fzattackbinseventeen
 \pgfplotstableread[col sep=comma]{results_bin/furuta_zero/a4_p0.25_alpha0.01_w20.csv}\fzattackbintwentyfive
 \pgfplotstableread[col sep=comma]{results_bin_chi2/furuta_zero/a4_p0.09_alpha0.01_w20.csv}\fzattackbinchinine
 \pgfplotstableread[col sep=comma]{results_bin_chi2/furuta_zero/a4_p0.17_alpha0.01_w20.csv}\fzattackbinchiseventeen
 \pgfplotstableread[col sep=comma]{results_bin_chi2/furuta_zero/a4_p0.25_alpha0.01_w20.csv}\fzattackbinchitwentyfive
 \pgfplotstableread[col sep=comma]{results_chi2/furuta_zero/a4_p0.09_alpha0.01_w20.csv}\fzattackchinine
 \pgfplotstableread[col sep=comma]{results_chi2/furuta_zero/a4_p0.17_alpha0.01_w20.csv}\fzattackchiseventeen
 \pgfplotstableread[col sep=comma]{results_chi2/furuta_zero/a4_p0.25_alpha0.01_w20.csv}\fzattackchitwentyfive 
 
 \nextgroupplot[ylabel style={align=center}, title = {Quadruple tank}, ymax=0.0037]
 \addplot[thick, blue, mark=*]
   table[ x expr=\thisrow{length}-2, restrict x to domain=0:25,y expr=(\thisrow{attack}-\thisrow{nominal})/4] {\attackbinnine};
 \addplot[thick, red, mark=*, mark options={solid, fill=white}]
   table[ x expr=\thisrow{length}-2, restrict x to domain=0:25, y expr=(\thisrow{attack}-\thisrow{nominal})/4]{\attackbinseventeen};
 \addplot[thick, brown, mark=x]
   table[ x expr=\thisrow{length}-2, restrict x to domain=0:25, y expr=(\thisrow{attack}-\thisrow{nominal})/4]{\attackbintwentyfive};
 \node[draw, fill=white, anchor=north east] at (axis cs:25, 0.0037){Binomial};

 \nextgroupplot[title= {Furuta pendulum}, ymax=0.00028]
 \addplot[thick, blue, mark=*, domain=0:25]
   table[ x expr=\thisrow{length}-2, restrict x to domain=0:25,y expr=(\thisrow{attack}-\thisrow{nominal})] {\fzattackbinnine};
 \addlegendentry{$\pmiss = 0.09$}
 \addplot[thick, red, mark=*, mark options={solid, fill=white}]
   table[ x expr=\thisrow{length}-2, restrict x to domain=0:25, y expr=(\thisrow{attack}-\thisrow{nominal})]{\fzattackbinseventeen};
 \addlegendentry{$\pmiss = 0.17$}
 \addplot[thick, brown, mark=x]
   table[ x expr=\thisrow{length}-2, restrict x to domain=0:25, y expr=(\thisrow{attack}-\thisrow{nominal})]{\fzattackbintwentyfive};
 \addlegendentry{$\pmiss = 0.25$}
 \node[draw, fill=white, anchor=north west] at (axis cs:1, 0.00028) {Binomial};

 \nextgroupplot[ymax=0.0037]
 \addplot[thick, blue, mark=*]
   table[ x expr=\thisrow{length}-2,restrict x to domain=0:25,y expr=(\thisrow{attack}-\thisrow{nominal})/4]{\attackbinchinine};
 \addplot[thick, red, mark=*, mark options={solid, fill=white}]
   table[ x expr=\thisrow{length}-2,restrict x to domain=0:25,y expr=(\thisrow{attack}-\thisrow{nominal})/4]{\attackbinchiseventeen};
 \addplot[thick, brown, mark=x]
   table[ x expr=\thisrow{length}-2,restrict x to domain=0:25,y expr=(\thisrow{attack}-\thisrow{nominal})/4]{\attackbinchitwentyfive};
 \node[draw, fill=white, anchor=north east] at (axis cs:25, 0.0037){Combined};

 \nextgroupplot[ymax=0.00028]
 \addplot[thick, blue, mark=*]
   table[ x expr=\thisrow{length}-2,restrict x to domain=0:25,y expr=(\thisrow{attack}-\thisrow{nominal})]{\fzattackbinchinine};
 \addplot[thick, red, mark=*, mark options={solid, fill=white}]
   table[ x expr=\thisrow{length}-2,restrict x to domain=0:25,y expr=(\thisrow{attack}-\thisrow{nominal})]{\fzattackbinchiseventeen};
 \addplot[thick, brown, mark=x]
   table[ x expr=\thisrow{length}-2,restrict x to domain=0:25,y expr=(\thisrow{attack}-\thisrow{nominal})]{\fzattackbinchitwentyfive};
 \node[draw, fill=white, anchor=north west] at (axis cs:1, 0.00028) {Combined};

 \nextgroupplot[ylabel style={align=center}, ymax=0.0037]
 \addplot[thick, blue, mark=*]
   table[ x expr=\thisrow{length}-2, restrict x to domain=0:25, y expr=(\thisrow{attack}-\thisrow{nominal})/4]{\attackchinine};
 \addplot[thick, red, mark=*, mark options={solid, fill=white}]
   table[ x expr=\thisrow{length}-2, restrict x to domain=0:25, y expr=(\thisrow{attack}-\thisrow{nominal})/4]{\attackchiseventeen};
 \addplot[thick, brown, mark=x]
   table[ x expr=\thisrow{length}-2, restrict x to domain=0:25, y expr=(\thisrow{attack}-\thisrow{nominal})/4]{\attackchitwentyfive};
 \node[draw, fill=white, anchor=north east] at (axis cs:25, 0.0037){Geometric};
 
 \nextgroupplot[ymax=0.00028]
 \addplot[thick, blue, mark=*]
   table[ x expr=\thisrow{length}-2, restrict x to domain=0:25, y expr=\thisrow{attack}-\thisrow{nominal}]{\fzattackchinine};
 \addplot[thick, red, mark=*, mark options={solid, fill=white}]
   table[ x expr=\thisrow{length}-2, restrict x to domain=0:25, y expr=\thisrow{attack}-\thisrow{nominal}]{\fzattackchiseventeen};
 \addplot[thick, brown, mark=x]
   table[ x expr=\thisrow{length}-2, restrict x to domain=0:25, y expr=\thisrow{attack}-\thisrow{nominal}]{\fzattackchitwentyfive};
 \node[draw, fill=white, anchor=north west] at (axis cs:1, 0.00028) {Geometric};
 
 \end{groupplot}
\end{tikzpicture}

    \vspace{-1cm}
    \caption{Attack effectiveness $\aeffect$ with different detection mechanisms: binomial test-based detector, geometric test-based detector and combined. The tests are conducted using the hold actuation policy for the quadruple tank and the zero actuation policy for the Furuta pendulum.}
    \vspace{-.35cm}
    \label{fig:effect_test}
\end{figure}

\begin{figure*}
    \centering
    \begin{tikzpicture} 
 \begin{groupplot} [%
 group style={columns=5, rows=2,
   xlabels at=edge bottom, ylabels at=edge left,
   xticklabels at=edge bottom, yticklabels at=edge left,
   horizontal sep=0.2cm, vertical sep=0.5cm},
 yticklabel style = {/pgf/number format/fixed, /pgf/number format/precision=5},
 scaled y ticks=false,
 %
 height=5cm, width = 0.25\textwidth,
 %
 ymin = 0,
 xmin = 1, xmax = 25,
 ylabel style={yshift=0cm},
 xlabel={$\alength$}, xlabel near ticks,
 grid=major,
 xtick = {5, 10, 15, 20, 25},
 grid style = {densely dashed, black!30},
 legend style={draw = none, at={(0.5,1.05)}, anchor=south, font=\small},
 legend columns = 3]

 % reading files for hold
 \pgfplotstableread[col sep=comma]{results_bin/furuta_hold/a4_p0.09_alpha0.01_w10.csv}\attackholdtennine
 \pgfplotstableread[col sep=comma]{results_bin/furuta_hold/a4_p0.17_alpha0.01_w10.csv}\attackholdtenseventeen
 \pgfplotstableread[col sep=comma]{results_bin/furuta_hold/a4_p0.25_alpha0.01_w10.csv}\attackholdtentwentyfive
 \pgfplotstableread[col sep=comma]{results_bin/furuta_hold/a4_p0.09_alpha0.01_w12.csv}\attackholdtwelvenine
 \pgfplotstableread[col sep=comma]{results_bin/furuta_hold/a4_p0.17_alpha0.01_w12.csv}\attackholdtwelveseventeen
 \pgfplotstableread[col sep=comma]{results_bin/furuta_hold/a4_p0.25_alpha0.01_w12.csv}\attackholdtwelvetwentyfive
 \pgfplotstableread[col sep=comma]{results_bin/furuta_hold/a4_p0.09_alpha0.01_w14.csv}\attackholdfourteennine
 \pgfplotstableread[col sep=comma]{results_bin/furuta_hold/a4_p0.17_alpha0.01_w14.csv}\attackholdfourteenseventeen
 \pgfplotstableread[col sep=comma]{results_bin/furuta_hold/a4_p0.25_alpha0.01_w14.csv}\attackholdfourteentwentyfive
 \pgfplotstableread[col sep=comma]{results_bin/furuta_hold/a4_p0.09_alpha0.01_w16.csv}\attackholdsixteennine
 \pgfplotstableread[col sep=comma]{results_bin/furuta_hold/a4_p0.17_alpha0.01_w16.csv}\attackholdsixteenseventeen
 \pgfplotstableread[col sep=comma]{results_bin/furuta_hold/a4_p0.25_alpha0.01_w16.csv}\attackholdsixteentwentyfive
 \pgfplotstableread[col sep=comma]{results_bin/furuta_hold/a4_p0.09_alpha0.01_w18.csv}\attackholdeighteennine
 \pgfplotstableread[col sep=comma]{results_bin/furuta_hold/a4_p0.17_alpha0.01_w18.csv}\attackholdeighteenseventeen
 \pgfplotstableread[col sep=comma]{results_bin/furuta_hold/a4_p0.25_alpha0.01_w18.csv}\attackholdeighteentwentyfive
 
 % reading files for pendulum zero
 \pgfplotstableread[col sep=comma]{results_bin/furuta_zero/a4_p0.09_alpha0.01_w10.csv}\attackzerotennine
 \pgfplotstableread[col sep=comma]{results_bin/furuta_zero/a4_p0.17_alpha0.01_w10.csv}\attackzerotenseventeen
 \pgfplotstableread[col sep=comma]{results_bin/furuta_zero/a4_p0.25_alpha0.01_w10.csv}\attackzerotentwentyfive
 \pgfplotstableread[col sep=comma]{results_bin/furuta_zero/a4_p0.09_alpha0.01_w12.csv}\attackzerotwelvenine
 \pgfplotstableread[col sep=comma]{results_bin/furuta_zero/a4_p0.17_alpha0.01_w12.csv}\attackzerotwelveseventeen
 \pgfplotstableread[col sep=comma]{results_bin/furuta_zero/a4_p0.25_alpha0.01_w12.csv}\attackzerotwelvetwentyfive
 \pgfplotstableread[col sep=comma]{results_bin/furuta_zero/a4_p0.09_alpha0.01_w14.csv}\attackzerofourteennine
 \pgfplotstableread[col sep=comma]{results_bin/furuta_zero/a4_p0.17_alpha0.01_w14.csv}\attackzerofourteenseventeen
 \pgfplotstableread[col sep=comma]{results_bin/furuta_zero/a4_p0.25_alpha0.01_w14.csv}\attackzerofourteentwentyfive
 \pgfplotstableread[col sep=comma]{results_bin/furuta_zero/a4_p0.09_alpha0.01_w16.csv}\attackzerosixteennine
 \pgfplotstableread[col sep=comma]{results_bin/furuta_zero/a4_p0.17_alpha0.01_w16.csv}\attackzerosixteenseventeen
 \pgfplotstableread[col sep=comma]{results_bin/furuta_zero/a4_p0.25_alpha0.01_w16.csv}\attackzerosixteentwentyfive
 \pgfplotstableread[col sep=comma]{results_bin/furuta_zero/a4_p0.09_alpha0.01_w18.csv}\attackzeroeighteennine
 \pgfplotstableread[col sep=comma]{results_bin/furuta_zero/a4_p0.17_alpha0.01_w18.csv}\attackzeroeighteenseventeen
 \pgfplotstableread[col sep=comma]{results_bin/furuta_zero/a4_p0.25_alpha0.01_w18.csv}\attackzeroeighteentwentyfive
 
 % PENDULUM ZERO -----------------------------------------------------------------------------------------------------------------------------------


    \nextgroupplot[ylabel style={align=center}, ylabel = {Zero, $\aeffect$}, ymax=0.00025]
    \addplot[thick, blue, mark=*] table[ x expr=\thisrow{length}-2, restrict x to domain=0:25, y expr=(\thisrow{attack}-\thisrow{nominal})]{\attackzerotennine};
    \addplot[thick, red, mark=*, mark options={solid, fill=white}] table[ x expr=\thisrow{length}-2, restrict x to domain=0:25, y expr=(\thisrow{attack}-\thisrow{nominal})]{\attackzerotenseventeen};
    \addplot[thick, brown, mark=x] table[ x expr=\thisrow{length}-2, restrict x to domain=0:25, y expr=(\thisrow{attack}-\thisrow{nominal})]{\attackzerotentwentyfive};
    \node[draw, fill=white, anchor=north west] at (axis cs:1,0.00025) {$\vecsize{\iteration}=10$};

    \nextgroupplot[ymax=0.00025]
    \addplot[thick, blue, mark=*] table[ x expr=\thisrow{length}-2, restrict x to domain=0:25, y expr=(\thisrow{attack}-\thisrow{nominal})]{\attackzerotwelvenine};
    \addplot[thick, red, mark=*, mark options={solid, fill=white}] table[ x expr=\thisrow{length}-2, restrict x to domain=0:25, y expr=(\thisrow{attack}-\thisrow{nominal})]{\attackzerotwelveseventeen};
    \addplot[thick, brown, mark=x] table[ x expr=\thisrow{length}-2, restrict x to domain=0:25, y expr=(\thisrow{attack}-\thisrow{nominal})]{\attackzerotwelvetwentyfive};
    \node[draw, fill=white, anchor=north west] at (axis cs:1,0.00025) {$\vecsize{\iteration}=12$};
  
    \nextgroupplot[ymax=0.00025]
    \addplot[thick, blue, mark=*] table[ x expr=\thisrow{length}-2,restrict x to domain=0:25,y expr=(\thisrow{attack}-\thisrow{nominal})]{\attackzerofourteennine};
    \addlegendentry{$\pmiss = 0.09$}
    \addplot[thick, red, mark=*, mark options={solid, fill=white}] table[ x expr=\thisrow{length}-2,restrict x to domain=0:25,y expr=(\thisrow{attack}-\thisrow{nominal})]{\attackzerofourteenseventeen};
    \addlegendentry{$\pmiss = 0.17$}
    \addplot[thick, brown, mark=x] table[ x expr=\thisrow{length}-2,restrict x to domain=0:25,y expr=(\thisrow{attack}-\thisrow{nominal})]{\attackzerofourteentwentyfive};
    \addlegendentry{$\pmiss = 0.25$}
    \node[draw, fill=white, anchor=north west] at (axis cs:1,0.00025) {$\vecsize{\iteration}=14$};
  
    \nextgroupplot[ymax=0.00025]
    \addplot[thick, blue, mark=*] table[ x expr=\thisrow{length}-2,restrict x to domain=0:25,y expr=(\thisrow{attack}-\thisrow{nominal})]{\attackzerosixteennine};
    \addplot[thick, red, mark=*, mark options={solid, fill=white}] table[ x expr=\thisrow{length}-2,restrict x to domain=0:25,y expr=(\thisrow{attack}-\thisrow{nominal})]{\attackzerosixteenseventeen};
    \addplot[thick, brown, mark=x] table[ x expr=\thisrow{length}-2,restrict x to domain=0:25,y expr=(\thisrow{attack}-\thisrow{nominal})]{\attackzerosixteentwentyfive};
    \node[draw, fill=white, anchor=north west] at (axis cs:1,0.00025) {$\vecsize{\iteration}=16$};
  
    \nextgroupplot[ymax=0.00025]
    \addplot[thick, blue, mark=*] table[ x expr=\thisrow{length}-2,restrict x to domain=0:25,y expr=(\thisrow{attack}-\thisrow{nominal})]{\attackzeroeighteennine};
    \addplot[thick, red, mark=*, mark options={solid, fill=white}] table[ x expr=\thisrow{length}-2,restrict x to domain=0:25,y expr=(\thisrow{attack}-\thisrow{nominal})]{\attackzeroeighteenseventeen};
    \addplot[thick, brown, mark=x] table[ x expr=\thisrow{length}-2,restrict x to domain=0:25,y expr=(\thisrow{attack}-\thisrow{nominal})]{\attackzeroeighteentwentyfive};
    \node[draw, fill=white, anchor=north west] at (axis cs:1,0.00025) {$\vecsize{\iteration}=18$};

 % PENDULUM HOLD -----------------------------------------------------------------------------------------------------------------------------------

    \nextgroupplot[ylabel style={align=center}, ylabel = {Hold, $\aeffect$}, ymax=0.15]
    \addplot[thick, blue, mark=*] table[ x expr=\thisrow{length}-2, restrict x to domain=0:25, y expr=(\thisrow{attack}-\thisrow{nominal})]{\attackholdtennine};
    \addplot[thick, red, mark=*, mark options={solid, fill=white}] table[ x expr=\thisrow{length}-2, restrict x to domain=0:25, y expr=(\thisrow{attack}-\thisrow{nominal})]{\attackholdtenseventeen};
    \addplot[thick, brown, mark=x] table[ x expr=\thisrow{length}-2, restrict x to domain=0:25, y expr=(\thisrow{attack}-\thisrow{nominal})]{\attackholdtentwentyfive};
    \node[draw, fill=white, anchor=north west] at (axis cs:1,0.15) {$\vecsize{\iteration}=10$};

    \nextgroupplot[ymax=0.15]
    \addplot[thick, blue, mark=*] table[ x expr=\thisrow{length}-2, restrict x to domain=0:25, y expr=(\thisrow{attack}-\thisrow{nominal})]{\attackholdtwelvenine};
    \addplot[thick, red, mark=*, mark options={solid, fill=white}] table[ x expr=\thisrow{length}-2, restrict x to domain=0:25, y expr=(\thisrow{attack}-\thisrow{nominal})]{\attackholdtwelveseventeen};
    \addplot[thick, brown, mark=x] table[ x expr=\thisrow{length}-2, restrict x to domain=0:25, y expr=(\thisrow{attack}-\thisrow{nominal})]{\attackholdtwelvetwentyfive};
    \node[draw, fill=white, anchor=north west] at (axis cs:1,0.15) {$\vecsize{\iteration}=12$};
  
    \nextgroupplot[ymax=0.15]
    \addplot[thick, blue, mark=*] table[ x expr=\thisrow{length}-2,restrict x to domain=0:25,y expr=(\thisrow{attack}-\thisrow{nominal})]{\attackholdfourteennine};
    \addplot[thick, red, mark=*, mark options={solid, fill=white}] table[ x expr=\thisrow{length}-2,restrict x to domain=0:25,y expr=(\thisrow{attack}-\thisrow{nominal})]{\attackholdfourteenseventeen};
    \addplot[thick, brown, mark=x] table[ x expr=\thisrow{length}-2,restrict x to domain=0:25,y expr=(\thisrow{attack}-\thisrow{nominal})]{\attackholdfourteentwentyfive};
    \node[draw, fill=white, anchor=north west] at (axis cs:1,0.15) {$\vecsize{\iteration}=14$};
  
    \nextgroupplot[ymax=0.15]
    \addplot[thick, blue, mark=*] table[ x expr=\thisrow{length}-2,restrict x to domain=0:25,y expr=(\thisrow{attack}-\thisrow{nominal})]{\attackholdsixteennine};
    \addplot[thick, red, mark=*, mark options={solid, fill=white}] table[ x expr=\thisrow{length}-2,restrict x to domain=0:25,y expr=(\thisrow{attack}-\thisrow{nominal})]{\attackholdsixteenseventeen};
    \addplot[thick, brown, mark=x] table[ x expr=\thisrow{length}-2,restrict x to domain=0:25,y expr=(\thisrow{attack}-\thisrow{nominal})]{\attackholdsixteentwentyfive};
    \node[draw, fill=white, anchor=north west] at (axis cs:1,0.15) {$\vecsize{\iteration}=16$};
  
    \nextgroupplot[ymax=0.15]
    \addplot[thick, blue, mark=*] table[ x expr=\thisrow{length}-2,restrict x to domain=0:25,y expr=(\thisrow{attack}-\thisrow{nominal})]{\attackholdeighteennine};
    \addplot[thick, red, mark=*, mark options={solid, fill=white}] table[ x expr=\thisrow{length}-2,restrict x to domain=0:25,y expr=(\thisrow{attack}-\thisrow{nominal})]{\attackholdeighteenseventeen};
    \addplot[thick, brown, mark=x] table[ x expr=\thisrow{length}-2,restrict x to domain=0:25,y expr=(\thisrow{attack}-\thisrow{nominal})]{\attackholdeighteentwentyfive};
    \node[draw, fill=white, anchor=north west] at (axis cs:1,0.15) {$\vecsize{\iteration}=18$};

 \end{groupplot}
\end{tikzpicture}

    \vspace{-1cm}
    \caption{Attack effectiveness $\aeffect$ for the Furuta pendulum with zero and hold actuation strategy in case of a deadline miss, varying the window size $\vecsize{\iteration}$, with $\maxa=4$ and using the binomial test-based attack detection mechanism.}
    %\vspace{-.35cm}
    \label{fig:effect_zerohold}
\end{figure*}

\end{document}